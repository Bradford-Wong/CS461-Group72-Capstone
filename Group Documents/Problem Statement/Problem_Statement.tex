\documentclass[onecolumn, draftclsnofoot,10pt, compsoc]{IEEEtran}
\usepackage{graphicx}
\usepackage{url}
\usepackage{setspace}

\usepackage{geometry}
\geometry{textheight=9.5in, textwidth=7in}

% 1. Fill in these details
%\def \CapstoneTeamName{		NotAvailableYet}
\def \CapstoneTeamNumber{		72}
\def \GroupMemberOne{			Brandon Jolly}
\def \GroupMemberTwo{			Katherine Jeffrey}
\def \GroupMemberThree{			Bradford Wong}
\def \CapstoneProjectName{		App to Support Field Diagnostics in Veterinary Medicine}
\def \CapstoneSponsorCompany{	Oregon State University}
\def \CapstoneSponsorPerson{		Dr. Christiane Loehr}

% 2. Uncomment the appropriate line below so that the document type works
\def \DocType{		Problem Statement
				%Requirements Document
				%Technology Review
				%Design Document
				%Progress Report
				}
			
\newcommand{\NameSigPair}[1]{\par
\makebox[2.75in][r]{#1} \hfil 	\makebox[3.25in]{\makebox[2.25in]{\hrulefill} \hfill		\makebox[.75in]{\hrulefill}}
\par\vspace{-12pt} \textit{\tiny\noindent
\makebox[2.75in]{} \hfil		\makebox[3.25in]{\makebox[2.25in][r]{Signature} \hfill	\makebox[.75in][r]{Date}}}}
%3. If the document is not to be signed, uncomment the RENEWcommand below
%\renewcommand{\NameSigPair}[1]{#1}

%%%%%%%%%%%%%%%%%%%%%%%%%%%%%%%%%%%%%%%
\begin{document}
\begin{titlepage}
    \pagenumbering{gobble}
    \begin{singlespace}
%    	\includegraphics[height=4cm]{coe_v_spot1}
        \hfill 
        % 4. If you have a logo, use this includegraphics command to put it on the coversheet.
        %\includegraphics[height=4cm]{CompanyLogo}   
        \par\vspace{.2in}
        \centering
        \scshape{
            \huge CS Capstone \DocType \par
            {\large\today}\par
            \vspace{.5in}
            \textbf{\Huge\CapstoneProjectName}\par
            \vfill
            {\large Prepared for}\par
            \Huge \CapstoneSponsorCompany\par
            \vspace{5pt}
            {\Large\NameSigPair{\CapstoneSponsorPerson}\par}
            {\large Prepared by }\par
            Group\CapstoneTeamNumber\par
            % 5. comment out the line below this one if you do not wish to name your team
 %          \CapstoneTeamName\par 
            \vspace{5pt}
            {\Large
                \NameSigPair{\GroupMemberOne}\par
                \NameSigPair{\GroupMemberTwo}\par
                \NameSigPair{\GroupMemberThree}\par
            }
            \vspace{20pt}
        }
        \begin{abstract}
        % 6. Fill in your abstract    
Currently, there are many difficulties with trying to perform remote diagnostics. There aren’t any effective ways for people out in the field collecting samples to communicate with specialized experts located in laboratories. As a result, this project will involve creating an android mobile application that will be used as a bridge to connect the field personnel with the veterinary pathologists in laboratories. With this mobile application, the field personnel will be able to take pictures of the individual that is being analyzed and then send the pictures along with other data such as the patient, location, and time to a pathologist. The pathologist will then be able to use the provided information to perform a necropsy and send feedback to the field personnel. This project is intended to improve remote field diagnostics in veterinary medicine.
        \end{abstract}     
    \end{singlespace}
\end{titlepage}
\newpage
\pagenumbering{arabic}
\tableofcontents
% 7. uncomment this (if applicable). Consider adding a page break.
%\listoffigures
%\listoftables
\clearpage

% 8. now you write!
\section{Definition and Description}
The Oregon Veterinary Diagnostic Laboratory (OVDL) sends field personnel to conduct remote diagnostics for animal diseases and perform necropsies. Necropsies, which are the autopsies of non-human species, are critical for a variety of reasons. These reasons include identifying diseases and cause of death, determining patterns of diseases among a group of individuals, and tracking how diseases develop over a specific area. The veterinary pathologists in the laboratories need to view the data and images collected at remote locations so they can make a preliminary diagnosis and decide if further action is needed. They need the data to be stored in a way that is accessible to both remote and lab personnel so they can easily find and refer to it.

The veterinarians in the lab need to be able to send messages to field personnel about the diagnostics and give feedback on data collected. They might need to request more information or give instructions regarding next steps in the diagnostic process. Some of the remote locations might not have a stable internet or cellular connection so the data will need to be stored until a connection can be established. Once the connection is made the data should be uploaded and the lab personnel should be alerted. 

When images are taken in the field they need to be high quality so the veterinarians in the lab can clearly see the important information they contain. Discoloration, shadows or reflections could distort the image and lead to a misdiagnosis or other problems. When a person in a remote location takes a picture of an autopsy they need to know right away if the image is clear enough to be sent to the lab. 

The data collected can be sensitive and needs to be protected so users will be required to give their credentials to access it. The field personnel should only be able to access the data they gathered and the information sent to them by the lab, not data gathered by other field personnel. There is a hierarchy of permissions for people working in the lab as well and they must only have access to the data they are allowed to see. 

The OVDL wants to test how effective remote diagnostics can be. To do this, the data sent from the field must be accurate and precise. Also, the method of sending data should not require much training to use in the field. If the tests are successful, then it could lead to an international spread of remote diagnostic work. 


\section{Proposed Solution}
The OVDL wants a native Android mobile app that collects field data and images, sends them to the database and gets real time feedback from the lab. When field personnel need to send reports to the lab, they can fill out a form on the app, which will have menus and prompts as well as a quota for images and text entry. The goal of this project is to deal with the disconnect and create an effective means of communication between the on-the-ground field personnel and the veterinary pathologists in laboratories. This mobile application will involve collecting image and data from necropsies performed at the Oregon Veterinary Diagnostic Laboratory in the College of Veterinary Medicine at Oregon State University’s campus and in the field.  

The data collected from the android application will be stored in a database architecture designed for a set of images that will be taken from an android device’s built-in camera. These images along with data including information such as patient, location, time, and additional text entries will be compiled into a report that users will be able to send to other users. Additionally, lab pathologists will be using this application as a way to communicate their feedback to the person in the field. This will improve the team’s ability to to identify lesions and make recommendations about collecting additional samples for more testing.  

Users will be able to use this android application in real time when they’re connected to the database and independently, which includes areas with limited phone or satellite service. Once the data is collected, it will be sent to a database as soon as an internet connection is established. The data will be stored in the phone until it can be transmitted. Ideally there will be a setting where users can decide to sync automatically when connected or send the data manually when they know they are connected. An interface will also be developed to link the application with the database. The database will be easily expandable and searchable. 

The data collected can be sensitive and needs to be protected so users will be required to give their credentials to access it. The field personnel should only be able to access the data they gathered and the information sent to them by the lab, not data gathered by other field personnel. There is a hierarchy of permissions for people working in the lab as well and they must only have access to the data they are allowed to see. Users will need to log in to the app to access stored information and to attach their credentials to any reports they send. This will give them permission to see previous reports and messages sent to them from the lab. They should be able to log out though the menu or settings screen. 

It is important to note that this project is meant to be a prototype. This means that the team may adopt additional new features to implement if it is feasible within the scope of the project. This could include implementing a feature of the application that would instantaneously provide feedback of the image taken and reject images with poor lighting, saturation and contrast. Again, since this project is intended to be a prototype, it is certainly possible that additional features like this one will not be delivered by the end of the project date.


\section{Performance Metrics}
	The project will be considered complete upon having a working prototype that contains certain required features.
    The app will allow the user to take a picture with an android device's camera and once the image is taken, the user will able to fill out additional information in drop-down menus.
	These drop-downs are used to reduce the amount of complexity involved in reporting an incident.
	Some options may be, “How many days have the animal had these symptom(s)." or, "How many animals have the same symptoms."
    
	In addition, users should be able to add comments in a text box to further describe the problem.
	Once the information is inputted, the user should then be able to tap a button to save the report on their phone.
    The app would then store the report in a database, which mirrors the one used in the lab.
	When the phone is connected to the internet, the app would ask the user if they wish to upload the information.
	If the user selects yes, then the application will send the information back to the lab to their database.
   	If the user selects no, then the application will wait until the user can send the message again and ask the user again. 
	
	To view the report the user inputted, the pathologist in the lab would use an API.
	This interface would be connected to database which stores the clients information and the reports they have submitted.
	They would then be able to group the information as they wish using filters on the interface.
    Once they finished reviewing the report the pathologist would respond to the user and send a replay back with any notes they would want to share.
    The database would then store this new information and then send the comments back to the user.
    An administrative account would be created for the client and their coworkers would only be able to modify reports their tasked with.  
    A notification would pop up informing the user they got feedback from their report.
   
 
\end{document}